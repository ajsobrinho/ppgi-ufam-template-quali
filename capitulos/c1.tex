%
% introdução
%
\chapter{Introdução}\label{sec:introducao}

...

\section{Problema}\label{sec:problem}

...

\section{Motivação}\label{sec:motivacao}

...

\section{Justificativa}\label{sec:justificativa}

...


%%%%%%%%%%%%%%%%%%%%%%%%%%%%%%%%%%%%%%%%%%%%%%%%%%%%%%%%%%%%%%%%%%%%%%%%%%%%%%%%
\section{Objetivo}\label{sec:objetivo}
Nesta seção serão apresentados os objetivos desta pesquisa.
\subsection{Objetivo Geral}

...

\begin{comment}
O objetivo dessa dissertação é desenvolver uma abordagem capaz de traduzir frases de LIBRAS para Português com resultados comparáveis ao estado da arte atual, através de técnicas de aprendizagem profunda.
\end{comment}

\subsection{Objetivos Específicos}
Para atingir o objetivo geral desta pesquisa, um conjunto de resultados intermediários deve ser alcançado:
\begin{enumerate}
\item ...
\begin{comment}
\item Gerar e disponibilizar uma base de vídeos de sinais de LIBRAS utilizados na alfabetização de surdos, devidamente rotulada em glosa e Português.
\item Adequar e combinar métodos de reconhecimento de sinal, a fim de alcançar um melhor desempenho do reconhecimento dos sinais.
\item Adequar e combinar métodos de tradução de codificação e decodificação, a fim de alcançar uma melhorar a semântica das frases geradas.

\end{comment}
\end{enumerate}


%%%%%%%%%%%%%%%%%%%%%%%%%%%%%%%%%%%%%%%%%%%%%%%%%%%%%%%%%%%%%%%%%%%%%%%%%%%%%%%%
\section{Contribuições Esperadas}\label{sec:contribuicao}

...

\section{Organização da proposta}\label{sec:organizacao}

No restante do documento está organizado da seguinte forma: o Capítulo~\ref{chap:fundamentacao} apresenta a fundamentação teórica que descreve os conceitos necessários para o entendimento dos aspectos gerais que compõem a abordagem proposta; o Capítulo~\ref{sec:trabalhos_relacionados} apresenta uma breve revisão da literatura recente sobre tradução de língua de sinal; o Capítulo~\ref{chap:abordagem_proposta} detalha a abordagem proposta neste trabalho; o Capítulo~\ref{sec:experimentos} apresenta a avaliação experimental da abordagem, bem como os resultados parciais obtidos; e por fim, o Capítulo~\ref{chap:conclusao} apresenta as conclusões parciais, limitações e trabalhos futuros.